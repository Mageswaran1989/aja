\chapter{Singular value decomposition (SVD)}

Singular value decomposition (SVD) factorizes a matrix into three matrices: U, \Sigma and V such.
For example Singular value decomposition takes a rectangular matrix of gene expression data (defined as A, where A is
a n x p matrix) in which the n rows represents the genes, and the p columns represents the experimental conditions.
The SVD theorem states:

A_{nxp} = U{nxn} \Sigma_{nxp} V^T{pxp},

where
U^TU = I_{nxn}
V^TV = I_{pxp}

- U is an orthonormal matrix, whose columns are called left singular vectors(gene coefficient vectors),
- \Sigma is a diagonal matrix with non-negative diagonals in descending order, whose diagonals are
   called singular values(mode amplitudes),
- V is an orthonormal matrix, whose columns are called right singular vectors (expression level vectors).

The SVD represents an expansion of the original data in a coordinate system where the covariance matrix is diagonal.
For large matrices, usually we don’t need the complete factorization but only the top singular values and its associated
singular vectors. This can save storage, de-noise and recover the low-rank structure of the matrix.

If we keep the top k singular values, then the dimensions of the resulting low-rank matrix will be:

U: m×k,
Σ: k×k,
V: n×k.


Calculating the SVD:
 Consists of finding the eigenvalues and eigenvectors of AA^T and A^TA.
 The eigenvectors of A^TA make up the columns of V.
 The eigenvectors of AA^T  make up the columns of U.
 Also, the singular values in S are square roots of eigenvalues from AA^T or A^TA.

 The singular values are the diagonal entries of the S matrix and are arranged in descending order.
 The singular values are always real numbers. If the matrix A is a real matrix, then U and V are also
real.

\section{Excercise}
Find SVD of A
A_{4,2} =
 \begin{bmatrix}
 2 & 4 \\
 1 & 3 \\
 0 & 0 \\
 0 & 00
 \end{bmatrix}


\subsection{Python}
import numpy as np
X = np.matrix("2,4; 1,3; 0,0; 0,0")
U, S, V = np.linalg.svd(X, full_matrices=False)
X_a = np.dot(np.dot(U, np.diag(S)), V)
print(X)
print(U)
print(S)
print(V)
print(X_a)
print(np.std(X), np.std(X_a), np.std(X - X_a))

\subsection{Scala}
Insert this
import breeze.linalg._
import breeze.numerics._
val a = new DenseMatrix(4,2, Array(2,1,0,0,4,3,0,0))
val svd.SVD(u,s,v) = svd(a)

\section{For more refer:}
- https://spark.apache.org/docs/latest/mllib-dimensionality-reduction.html
- http://web.mit.edu/be.400/www/SVD/Singular_Value_Decomposition.htm
