\chapter{Eigenvalues and eigenvectors}

In linear algebra, an eigenvector or characteristic vector of a square matrix is a vector that does not change its
direction under the associated linear transformation. In other words—if \mathbf{v} is a vector that is not zero, then it
is an eigenvector of a square matrix A if Av is a scalar multiple of v. This condition could be written as the equation

A\mathbf{v} = \lambda \mathbf{v},

 where \lambda is a number (also called a scalar) known as the eigenvalue or characteristic value associated with the
 eigenvector v. Geometrically, an eigenvector corresponding to a real, nonzero eigenvalue points in a direction that
 is stretched by the transformation and the eigenvalue is the factor by which it is stretched. If the eigenvalue is
 negative, the direction is reversed


How to find \mathbf{v}:

A\mathbf{v} - \lambda \mathbf{v} = 0
(A-\lambda I)\mathbf{v} = 0
which is of form
Ax = b, that is find vector x!

 \section{For more Refer:}
 https://en.wikipedia.org/wiki/Eigenvalues_and_eigenvectors