\section{Neural Processing}
How do you recognize a face in a crowd? How does an economist predict the
direction of interest rates? Faced with problems like these, the human brain
uses a web of interconnected processing elements called neurons to process
information. Each neuron is autonomous and independent; it does its work
asynchronously, that is, without any synchronization to other events taking
place. The two problems posed, namely recognizing a face and forecasting
interest rates, have two important characteristics that distinguish them from
other problems: First, the problems are complex, that is, you can’t devise a
simple step-by-step algorithm or precise formula to give you an answer; and
second, the data provided to resolve the problems is equally complex and may
be noisy or incomplete. You could have forgotten your glasses when you’re
trying to recognize that face. The economist may have at his or her disposal
thousands of pieces of data that may or may not be relevant to his or her
forecast on the economy and on interest rates. \\*
The vast processing power inherent in biological neural structures has inspired
the study of the structure itself for hints on organizing human-made computing
structures. Artificial neural networks, the subject of this book, covers the way
to organize synthetic neurons to solve the same kind of difficult, complex
problems in a similar manner as we think the human brain may. This chapter
will give you a sampling of the terms and nomenclature used to talk about
neural networks. These terms will be covered in more depth in the chapters to
follow.
