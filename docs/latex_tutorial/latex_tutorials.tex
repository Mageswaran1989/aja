%Frequently update this section from aja/docs/latex_tutorial/latex_template.tex
%To get all latest inclusion updates on book seetings.

\documentclass[12pt, right open]{memoir}
\usepackage{graphicx}
\usepackage{tikz}
\usepackage{xcolor}
\usetikzlibrary{matrix,chains,positioning,decorations.pathreplacing,arrows,automata}
\usetikzlibrary{shapes.geometric, calc, intersections}
\usepackage{mathtools}
\usepackage{amsmath}
\usepackage{float}
\floatstyle{boxed}
\restylefloat{figure}
\usepackage{multirow} %for tables
\usepackage{ifthen}
\setcounter{secnumdepth}{5}

%To reduce vetical line spaces between list items
\usepackage{enumitem}
\setlist{nolistsep,leftmargin=*}

\newcommand{\specialcell}[2][c]{%
  \begin{tabular}[#1]{@{}c@{}}#2\end{tabular}}
%  Foo bar & \specialcell{Foo\\bar} & Foo bar \\    % vertically centered
%Foo bar & \specialcell[t]{Foo\\bar} & Foo bar \\ % aligned with top rule
%Foo bar & \specialcell[b]{Foo\\bar} & Foo bar \\ % aligned with bottom rule

\newcommand{\matplus}{
~~
  }

%For C++ code
\usepackage{listings}
\usepackage{xcolor} % for setting colors

% set the default code style
\lstset{
    frame=tb, % draw a frame at the top and bottom of the code block
    tabsize=4, % tab space width
    showstringspaces=false, % don't mark spaces in strings
 %   numbers=left, % display line numbers on the left
    commentstyle=\color{green}, % comment color
    keywordstyle=\color{blue}, % keyword color
    stringstyle=\color{red} % string color
    breaklines=true,
}

\lstdefinestyle{codeTex}{
  belowcaptionskip=1\baselineskip,
  breaklines=true,
  frame=L,
  xleftmargin=\parindent,
  language=Tex,
  showstringspaces=false,
  basicstyle=\footnotesize\ttfamily,
  keywordstyle=\bfseries\color{green!40!black},
  commentstyle=\itshape\color{purple!40!black},
  identifierstyle=\color{blue},
  stringstyle=\color{orange},
}
%\begin{lstlisting}
%---------------
%\end{lstlisting}
%\lstinputlisting[caption=Scheduler, style=customc]{hello.c}


\begin{document}

%http://www.comp.leeds.ac.uk/ai23/reading/Hopfield.pdf

% >>>>>>>> Tikz Style sheet
\tikzstyle{every pin edge}=[<-,shorten <=1pt]
\tikzstyle{neuron}=[circle,fill=black!25,minimum size=17pt,inner sep=0pt]
\tikzstyle{input neuron}=[neuron, fill=black!50]
\tikzstyle{output neuron}=[neuron, fill=black!50]
\tikzstyle{hidden neuron}=[neuron, fill=black!50]
\tikzstyle{annot} = [text width=4em, text centered]

\tikzstyle{startstop} = [rectangle, rounded corners, minimum width=3cm, minimum height=1cm,text centered, draw=black]
\tikzstyle{io} = [trapezium, trapezium left angle=70, trapezium right angle=110, minimum width=3cm, minimum height=1cm, text centered, draw=black]
\tikzstyle{process} = [rectangle, minimum width=3cm, minimum height=1cm, text centered, text width=3cm, draw=black]
\tikzstyle{decision} = [diamond, minimum width=3cm, minimum height=1cm, text centered, draw=black]
\tikzstyle{arrow} = [thick,->,>=stealth]
% <<<<<<<< Tikz Style sheet

%%%%%%%%%%%%%%%%%%%%%%%%%%%%%%%%%%%%%%%%%%%%%%%%%%%%%%%%%%%%%%%%%%%%%%%%%%%%%%%%%%%%%%%%%%%
\chapter{References for complex Latex commands}
%Sub heading in the chapter
\section{Basic Flowchart}

All flowcharts related latex commands can be followed here.
 
Reference: https://www.sharelatex.com/blog/2013/08/29/tikz-series-pt3.html

% Create Tikz style, something like typedef in C, where we can specify the shape, color, size, text details etc

\tikzstyle{startstop} = [rectangle, rounded corners, minimum width=3cm, minimum height=1cm,text centered, draw=black, fill=red!30]
\tikzstyle{io} = [trapezium, trapezium left angle=70, trapezium right angle=110, minimum width=3cm, minimum height=1cm, text centered, draw=black, fill=blue!30]
%\tikzstyle{process} = [rectangle, minimum width=3cm, minimum height=1cm, text centered, draw=black, fill=orange!30]
\tikzstyle{process} = [rectangle, minimum width=3cm, minimum height=1cm, text centered, text width=3cm, draw=black, fill=orange!30]
\tikzstyle{decision} = [diamond, minimum width=3cm, minimum height=1cm, text centered, draw=black, fill=green!30]
\tikzstyle{arrow} = [thick,->,>=stealth]

\begin{figure}[h!] %Create figure holder
\begin{tikzpicture}[node distance=2cm] %use the ‘tikzpicture’ environment

% Nodes are very powerful as we can easily position them, make them draw a shape, heavily format them and give them some text. In square brackets at the end of the begin command we specify a node distance of 2cm. This is so that the nodes we use to build the blocks are automatically spaced 2cm apart from their centres.

%      node_var  style   display text
\node (start) [startstop] {Start};
\node (in1) [io, below of=start] {Input};
\node (pro1) [process, below of=in1] {Process 1};
\node (dec1) [decision, below of=pro1] {Decision 1};
\node (dec1) [decision, below of=pro1, yshift=-0.5cm] {Decision 1};
\node (pro2a) [process, below of=dec1, yshift=-0.5cm] {Process 2a};
\node (pro2b) [process, right of=dec1, xshift=2cm] {Process 2b};
\node (out1) [io, below of=pro2a] {Output};
\node (stop) [startstop, below of=out1] {Stop};

\draw [arrow] (start) -- (in1);
\draw [arrow] (in1) -- (pro1);
\draw [arrow] (pro1) -- (dec1);
\draw [arrow] (dec1) -- (pro2a);
\draw [arrow] (dec1) -- (pro2b);
\draw [arrow] (dec1) -- node[anchor=east] {yes} (pro2a);
\draw [arrow] (dec1) -- node[anchor=south] {no} (pro2b);

\draw [arrow] (pro2b) |- (pro1);
\draw [arrow] (pro2a) -- (out1);
\draw [arrow] (out1) -- (stop);

\node (pro2a) [process, below of=dec1, yshift=-0.5cm] {Process 2a text text text text text text text text text text};

\end{tikzpicture}
\end{figure}

\begin{lstlisting}[style=codeTex]
% Create Tikz style, something like typedef in C, 
where we can specify the shape, color, size, text details etc

\tikzstyle{startstop} = [rectangle, rounded corners, minimum width=3cm, 
minimum height=1cm,text centered, draw=black, fill=red!30]
\tikzstyle{io} = [trapezium, trapezium left angle=70, trapezium right angle=110, minimum width=3cm, minimum height=1cm, text centered, draw=black, fill=blue!30]
%\tikzstyle{process} = [rectangle, minimum width=3cm, minimum height=1cm, text centered, draw=black, fill=orange!30]
\tikzstyle{process} = [rectangle, minimum width=3cm, minimum height=1cm, text centered, text width=3cm, draw=black, fill=orange!30]
\tikzstyle{decision} = [diamond, minimum width=3cm, minimum height=1cm, text centered, draw=black, fill=green!30]
\tikzstyle{arrow} = [thick,->,>=stealth]

\begin{figure}[h!] %Create figure holder
\begin{tikzpicture}[node distance=2cm] %use the ‘tikzpicture’ environment

% Nodes are very powerful as we can easily position them, make them draw a shape, heavily format them and give them some text. In square brackets at the end of the begin command we specify a node distance of 2cm. This is so that the nodes we use to build the blocks are automatically spaced 2cm apart from their centres.

%      node_var  style   display text
\node (start) [startstop] {Start};
\node (in1) [io, below of=start] {Input};
\node (pro1) [process, below of=in1] {Process 1};
\node (dec1) [decision, below of=pro1] {Decision 1};
\node (dec1) [decision, below of=pro1, yshift=-0.5cm] {Decision 1};
\node (pro2a) [process, below of=dec1, yshift=-0.5cm] {Process 2a};
\node (pro2b) [process, right of=dec1, xshift=2cm] {Process 2b};
\node (out1) [io, below of=pro2a] {Output};
\node (stop) [startstop, below of=out1] {Stop};

\draw [arrow] (start) -- (in1);
\draw [arrow] (in1) -- (pro1);
\draw [arrow] (pro1) -- (dec1);
\draw [arrow] (dec1) -- (pro2a);
\draw [arrow] (dec1) -- (pro2b);
\draw [arrow] (dec1) -- node[anchor=east] {yes} (pro2a);
\draw [arrow] (dec1) -- node[anchor=south] {no} (pro2b);

\draw [arrow] (pro2b) |- (pro1);
\draw [arrow] (pro2a) -- (out1);
\draw [arrow] (out1) -- (stop);

\node (pro2a) [process, below of=dec1, yshift=-0.5cm] {Process 2a text text text text text text text text text text};

\end{tikzpicture}
\end{figure}
\end{lstlisting}
%\lstinputlisting[caption=FlowChart, style=codeTex]{}
\end{document}
