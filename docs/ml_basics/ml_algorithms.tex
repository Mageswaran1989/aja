%Frequently update this section from aja/docs/latex_tutorial/latex_template.tex
%To get all latest inclusion updates on book seetings.

%TODO: This should be made as book template for aja use!
%memoir  is the book template from LaTeX
\documentclass[12pt, right open]{memoir}
%To draw stuff on out documents
\usepackage{graphicx}
\usepackage{tikz}
\usepackage{xcolor}
%indivudual library needed on ad-hoc basis
\usetikzlibrary{matrix,chains,positioning,decorations.pathreplacing,arrows,automata}
%To perform coordinate calculations, the calc library is required,
%calculations are enclosed in $
\usetikzlibrary{shapes.geometric, calc, intersections}
\usepackage{pgfplots}
%Maths tools
\usepackage{mathtools}
\usepackage{amsmath}
\usepackage{float}
\floatstyle{boxed}
\restylefloat{figure}
\usepackage{multirow} %for tables
\usepackage{ifthen}
%http://en.wikibooks.org/wiki/LaTeX/Algorithms
\usepackage{algorithmic}
\setcounter{secnumdepth}{5}

%To reduce vetical line spaces between list items
\usepackage{enumitem}
\setlist{nolistsep,leftmargin=*}

\newcommand{\specialcell}[2][c]{%
  \begin{tabular}[#1]{@{}c@{}}#2\end{tabular}}
%  Foo bar & \specialcell{Foo\\bar} & Foo bar \\    % vertically centered
%Foo bar & \specialcell[t]{Foo\\bar} & Foo bar \\ % aligned with top rule
%Foo bar & \specialcell[b]{Foo\\bar} & Foo bar \\ % aligned with bottom rule

%to tackle spaces in -1 n +1
\newcommand{\matplus}{
~~
  }

%For listing code
\usepackage{listings}
\usepackage{xcolor} % for setting colors

% set the default code style
\lstset{
    frame=tb, % draw a frame at the top and bottom of the code block
    tabsize=4, % tab space width
    showstringspaces=false, % don't mark spaces in strings
 %   numbers=left, % display line numbers on the left
    commentstyle=\color{green}, % comment color
    keywordstyle=\color{blue}, % keyword color
    stringstyle=\color{red} % string color
    breaklines=true,
}

\lstdefinestyle{codeTex}{
  belowcaptionskip=1\baselineskip,
  breaklines=true,
  frame=L,
  xleftmargin=\parindent,
  language=Tex,
  showstringspaces=false,
  basicstyle=\footnotesize\ttfamily,
  keywordstyle=\bfseries\color{green!40!black},
  commentstyle=\itshape\color{purple!40!black},
  identifierstyle=\color{blue},
  stringstyle=\color{orange},
}

\lstdefinestyle{codeC}{
  belowcaptionskip=1\baselineskip,
  breaklines=true,
  frame=L,
  xleftmargin=\parindent,
  language=C,
  showstringspaces=false,
  basicstyle=\footnotesize\ttfamily,
  keywordstyle=\bfseries\color{green!40!black},
  commentstyle=\itshape\color{purple!40!black},
  identifierstyle=\color{blue},
  stringstyle=\color{orange},
}
%\begin{lstlisting}[style=codeC]
%---------------
%\end{lstlisting}
%\lstinputlisting[caption=Scheduler, style=customc]{hello.c}


\begin{document}

%http://www.comp.leeds.ac.uk/ai23/reading/Hopfield.pdf

%TODO: Needs to find an easy way to draw neural nets

% >>>>>>>> Tikz Style sheet
% Create Tikz style, something like typedef in C, where we can specify the shape, color, size, text details etc

\tikzstyle{every pin edge}=[<-,shorten <=1pt]
\tikzstyle{neuron}=[circle,fill=black!10,minimum size=25pt,inner sep=0pt]
\tikzstyle{input neuron}=[neuron, fill=black!40]
\tikzstyle{output neuron}=[neuron, fill=black!40]
\tikzstyle{hidden neuron}=[neuron, fill=black!10]
\tikzstyle{annot} = [text width=4em, text centered]

\tikzstyle{startstop} = [rectangle, rounded corners, minimum width=3cm, minimum height=1cm,text centered, draw=black, fill=red!30]
\tikzstyle{io} = [trapezium, trapezium left angle=70, trapezium right angle=110, minimum width=3cm, minimum height=1cm, text centered, draw=black, fill=blue!30]
%\tikzstyle{process} = [rectangle, minimum width=3cm, minimum height=1cm, text centered, draw=black, fill=orange!30]
\tikzstyle{process} = [rectangle, minimum width=3cm, minimum height=1cm, text centered, text width=3cm, draw=black, fill=orange!30]
\tikzstyle{decision} = [diamond, minimum width=3cm, minimum height=1cm, text centered, draw=black, fill=green!30]
\tikzstyle{arrow} = [thick,->,>=stealth]
% <<<<<<<< Tikz Style sheet

\tableofcontents

%%%%%%%%%%%%%%%%%%%%%%%%%%%%%%%%%%%%%%%%%%%%%%%%%%%%%%%%%%%%%%%%%%%%%%%%%%%%%%%%%%%%%%%%%%%
\chapter{Machine Learning Algorithms}

\section{The Big List}
\begin{itemize}
\item Decision tree Induction
\item Nearest Neighbour

\end{itemize}

\end{document}
