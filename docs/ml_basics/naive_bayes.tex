%Frequently update this section from aja/docs/latex_tutorial/latex_template.tex
%To get all latest inclusion updates on book seetings.

%TODO: This should be made as book template for aja use!
\documentclass[12pt, right open]{memoir}
\usepackage{graphicx}
\usepackage{tikz}
\usepackage{xcolor}
\usetikzlibrary{matrix,chains,positioning,decorations.pathreplacing,arrows,automata}
\usetikzlibrary{shapes.geometric, calc, intersections}
\usepackage{mathtools}
\usepackage{amsmath}
\usepackage{float}
\floatstyle{boxed}
\restylefloat{figure}
\usepackage{multirow}
\usepackage{ifthen}
\usepackage{algorithmic}
\setcounter{secnumdepth}{5}
\usepackage{enumitem}
\setlist{nolistsep,leftmargin=*}
\usepackage{listings}
\usepackage{xcolor}
\usepackage{hyperref}

\newcommand{\specialcell}[2][c]
{
  \begin{tabular}[#1]{@{}c@{}}#2\end{tabular}
}
\newcommand{\matplus}
{
~~
}
\newcommand{\matplus}
{
~~
}
\lstset{
    frame=tb, 
    tabsize=4, 
    showstringspaces=false, 
 %   numbers=left, % display line numbers on the left
    commentstyle=\color{green}, 
    keywordstyle=\color{blue}, 
    stringstyle=\color{red},
    breaklines=true,
}

\lstdefinestyle{codeTex}{
  belowcaptionskip=1\baselineskip,
  breaklines=true,
  frame=L,
  xleftmargin=\parindent,
  language=Tex,
  showstringspaces=false,
  basicstyle=\footnotesize\ttfamily,
  keywordstyle=\bfseries\color{green!40!black},
  commentstyle=\itshape\color{purple!40!black},
  identifierstyle=\color{blue},
  stringstyle=\color{orange},
}

\lstdefinestyle{codeC}{
  belowcaptionskip=1\baselineskip,
  breaklines=true,
  frame=L,
  xleftmargin=\parindent,
  language=C,
  showstringspaces=false,
  basicstyle=\footnotesize\ttfamily,
  keywordstyle=\bfseries\color{green!40!black},
  commentstyle=\itshape\color{purple!40!black},
  identifierstyle=\color{blue},
  stringstyle=\color{orange},
}
\begin{document}
%TODO: Needs to find an easy way to draw neural nets
\tikzstyle{every pin edge}=[<-,shorten <=1pt]
\tikzstyle{neuron}=[circle,fill=black!10,minimum size=25pt,inner sep=0pt]
\tikzstyle{input neuron}=[neuron, fill=black!40]
\tikzstyle{output neuron}=[neuron, fill=black!40]
\tikzstyle{hidden neuron}=[neuron, fill=black!10]
\tikzstyle{annot} = [text width=4em, text centered]

\tikzstyle{startstop} = [rectangle, rounded corners, minimum width=3cm, minimum height=1cm,text centered, draw=black, fill=red!30]
\tikzstyle{io} = [trapezium, trapezium left angle=70, trapezium right angle=110, minimum width=3cm, minimum height=1cm, text centered, draw=black, fill=blue!30]
\tikzstyle{process} = [rectangle, minimum width=3cm, minimum height=1cm, text centered, text width=3cm, draw=black, fill=orange!30]
\tikzstyle{decision} = [diamond, minimum width=3cm, minimum height=1cm, text centered, draw=black, fill=green!30]
\tikzstyle{arrow} = [thick,->,>=stealth]


\tableofcontents

%%%%%%%%%%%%%%%%%%%%%%%%%%%%%%%%%%%%%%%%%%%%%%%%%%%%%%%%%%%%%%%%%%%%%%%%%%%%%%%%%%%%%%%%%%%

\chapter{Bayesian Statistics}
Bayesian statistics is a collection of tools that is used in a special form of statistical
inference which applies in the analysis of experimental data in many practical
situations in science and engineering. Bayes’ rule is one of the most important
rules in probability theory.


\section{YYY}

\end{document}